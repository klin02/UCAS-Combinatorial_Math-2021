\documentclass[fontset=windows,11pt]{article}
\usepackage[a4paper]{geometry}
\geometry{left=2.0cm,right=2.0cm,top=2.5cm,bottom=2.5cm}

\usepackage[UTF8]{ctex}
\usepackage{amsmath,amsfonts,graphicx,amssymb,bm,amsthm}
\usepackage{algorithm}
\usepackage{algorithmicx}
\usepackage[noend]{algpseudocode}
\usepackage{mathtools}

\newtheorem{question}{\hskip 1.7em}

\newenvironment{solution}{{\noindent\hskip 2em \bf 解 \quad}}

\renewenvironment{proof}{{\noindent\hskip 2em \bf 证明 \quad}}{\hfill$\qed$\par\vskip1em}

\begin{document}
    \begin{center}
        {\Large \bf 组合数学10.04思考题}\\
    \end{center}
    \begin{kaishu}
        \hfill 提交者:游昆霖 \quad 学号:2020K8009926006
    \end{kaishu}

\begin{question}
    n*n方格图用1*2或2*1的多米诺牌进行覆盖,总共有多少种可能性?请给出显式表达式或估计。
\end{question}
\begin{solution}
    显然,当且仅当n为偶数的时候有非零的多米诺牌覆盖情形数。一般地,考虑$m*n$方格图(m,n均为偶数)的多米诺牌覆盖情形。\par 
    不妨用每个方格中心点代表该方格,用连接水平或垂直相邻点的线代表多米诺牌,则原问题可转化为如下形式:\par 
    考虑一个$m*n$的点阵(m,n均为偶数),规定一个点作为且仅作为一条边的端点,所有边均为水平或竖直的,求满足以上条件的边的排布方式。\par
    由于所有的边都是相邻点的连线,可以考虑将该点阵二染色,使得相邻点异色(例如,染色如黑白棋盘),则该点阵可转化为二分图,且黑点数=白点数
    =$\frac{mn}{2}$,此时,每种满足条件的边的排布方式(方便起见,以下称为可行排布)对应二分图的一种完美匹配(反之不成立,此因黑点和白点不是任意相邻,不可任意连接) \par 
    对黑点和白点,分别从左上角向右下角逐行编码,记为$a_i,b_i\quad (i=1,2,\cdots,\frac{mn}{2})$,则一种可行排布对应一个置换$\sigma$,使得$a_i$与$b_{\sigma(i)}$相邻,用$\sim $表示相邻,则只需求
    \[F_{m,n}=\sum_{\sigma,a_i\sim b_{\sigma(i)},\forall i}1\]\par 
    已知$\frac{mn}{2}\times \frac{mn}{2}$矩阵行列式可表达为$det K=\sum_\sigma sgn(\sigma)k_{1,\sigma(1)}\cdots k_{\frac{mn}{2},\sigma(\frac{mn}{2})}$\par 
    注意到:$sgn(\sigma)$可表示为$(-1)^k$,其中k为该置换可分解的对换个数。观察得到,若一个可行排布复合奇数次对换仍为一个可行排布,则必将增加2条竖直边(在mod4的意义下),且符号乘-1,
    考虑到最平凡的可行分布$\sigma_0:i\rightarrow i$,符号为1,可为表示竖直边的元素赋权i以消去$\sigma$符号的改变,为使以上两式有相同形式,我们可对K的元素进行以下赋值
    \begin{align*}
        k_{i,j}=\begin{cases}
            1\quad a_i\sim b_j,\text{且连线水平}\\
            i\quad a_i\sim b_j,\text{且连线竖直}\\
            0\quad \text{else}
        \end{cases}
    \end{align*}\par 
    此时,K即为该点阵的邻接矩阵(其中,表示竖直边的元素替换为i以消去$sgn(\sigma)$),不妨将该矩阵称为点阵的二部邻接矩阵,则只需求$det K$\par 
    对该点阵重新编码,$s.t.\quad a_i=c_i,b_i=c_{i+\frac{mn}{2}}\forall i=1,\cdots,\frac{mn}{2}$此时$mn\times mn$矩阵$C_{m,n}=\begin{bmatrix*}
        0,K\\
        K^t,0
    \end{bmatrix*}$即为该点阵的邻接矩阵(在将权重i考虑为1的情况下),且$det C_{m,n}=(det K)^2$\par 
    ~\par 
    邻接矩阵的行列式与编码顺序无关,且点阵图G是长度为m的直线点列与长度为n的直线点列的笛卡尔积。记这两个点列的邻接矩阵分别为$C_m,C_n$,且$C_m,C_n$的特征值和特征向量
    分别为$\lambda_m,\vec{v_m}$与$\lambda_n,\vec{v_n}$,则$C_{m,n}=C_m\otimes I_n+iI_m\otimes C_n,\quad C_{m,n}$的特征值为$\lambda_m+i\lambda_n$;[注记1]\par 
    ~\par 
    注意到$C_m=J_m+J_m^{-1}$其中$J_m$为m阶(上)若尔当块,其逆为m阶下若尔当块(不考虑边界的情况下),方便起见,先计算无穷阶若尔当块的特征值$\mu$,则$\lambda_m=\mu+\mu^{-1}$\par 
    由$C_m$拓展到无穷阶的特征值可知,$C_m$的特征向量必为$\vec{w_1}=[\cdots,\mu^{-1},1,\mu^1,\cdots]$和$\vec{w_2}=[\cdots,\mu^1,1,\mu^{-1},\cdots]$的线性组合。同时考虑边界条件,
    若$\vec{v}=[v_1,\cdots,v_m]$为$C_m$的特征向量,则应满足$v_0=v_{m+1}=0$,注意到$\vec{v_1}-\vec{v_2}$可出现0项,将其作为$v_0$,则同时应当满足$v_{m+1}=\mu^{m+1}-\mu^{-m-1}=0$,
    故解得$\mu=e^{i\pi /m+1},\quad\lambda_m=2cos\frac{i\pi}{m+1}\quad (i=1,\cdots,m)$,完全同理,有$\lambda_n=2cos\frac{j\pi}{m+1}\quad (j=1,\cdots,n)$\par 
    由特征多项式及多元韦达定理易知,矩阵的行列式为所有特征值(计算重数)的乘积,于是我们可以得到以下式子:
    \begin{align*}
        det K&= \sqrt{det C_{m,n}}\\
        &=\prod_{i=1}^m\prod_{j=1}^n\sqrt{2cos\frac{i\pi}{m+1}+2icos\frac{j\pi}{n+1}} 
    \end{align*}\par
    该结果即为所求。
    ~\par 
    ~\par 
    ~\par 
    ~\par 
    [注记1]:笔者离散数学与线性代数课程对部分相关知识进行了介绍,但未深入学习图的笛卡尔积与对应邻接矩阵关系式,以及相关张量知识,本自然段内容通过网上查阅代数图论与张量相关文献获得,此处不加证明的加以引用。
    
\end{solution}
\end{document}