\documentclass[11pt]{article}

\usepackage[a4paper]{geometry}
\geometry{left=2.0cm,right=2.0cm,top=2.5cm,bottom=2.5cm}

\usepackage[UTF8]{ctex}
\usepackage{amsmath,amsfonts,graphicx,amssymb,bm,amsthm}
\usepackage{algorithm,algorithmicx}
\usepackage[noend]{algpseudocode}
\usepackage{fancyhdr}

\usepackage{mathtools}
\usepackage{hyperref}

\newtheorem{theorem}{{\hskip 1.7em \bf 定理}}
\newtheorem{lemma}[theorem]{\hskip 1.7em 引理}
\newtheorem{proposition}[theorem]{Proposition}
\newtheorem{claim}[theorem]{\hskip 1.7em 命题}
\newtheorem{corollary}[theorem]{\hskip 1.7em 推论}
\newtheorem{definition}[theorem]{\hskip 1.7em 定义}
\newtheorem{example}{\hskip 1.7em 例}

\renewcommand{\emph}[1]{\begin{kaishu} #1 \end{kaishu}}

\newenvironment{solution}{{\noindent\hskip 2em \bf 解 \quad}}


\renewenvironment{proof}{{\noindent\hskip 2em \bf 证明 \quad}}{\hfill$\qed$\par\vskip1em}
\newenvironment{proofs}[1]{{\noindent\hskip 2em \bf #1 \quad}}{\hfill$\qed$\par\vskip1em}

\newenvironment{concept}[1]{{\bf #1\quad} \begin{kaishu}} {\end{kaishu}\par}

\newcommand\E{\mathbb{E}}

\begin{document}

    \pagestyle{fancy}
    \lhead{\kaishu 中国科学院大学}
    \chead{}
    \rhead{\kaishu 2021年秋季学期组合数学}

    \begin{center}
        {\LARGE \bf 组合数学第六讲}\\
    \end{center}
        \begin{kaishu}
            授课时间: 2021年10月4日\quad
            授课教师: 孙晓明
            \hfill 记录人: 林盛昊\quad 游昆霖\quad 洪泽润
        \end{kaishu}
        
\section{回顾与拓展(生成函数)} 
    任给数列 $a_0,a_1,a_2,\ldots$它对应的生成函数为$A(x)=a_0+a_1x+a_2x^2+\ldots$\par 
    例如,有以下数列和生成函数的对应关系
\begin{align*}
    &1,1,1,1,\ldots  \rightarrow 1+x+x^2+x^3+\ldots=\frac{1}{1-x}\\
    &1,2,3,4,\ldots\rightarrow 1+2x+3x^2+4x^3+\ldots=(\frac{1}{1-x})'=\frac{1}{{(1-x)}^2}\\
    &0,1,2,3,\ldots\rightarrow 0+x+2x^2+3x^3+\ldots=\frac{x}{{(1-x)}^2}\\
    &1,2,4,8,\ldots\rightarrow 1+2x+4x^2+8x^3+\ldots=\frac{1}{1-2x}
\end{align*}    

\begin{example}
汉诺塔(Tower of Hanoi)    
\end{example}
根据递推关系$h(n)=2h(n-1)+1\quad (n\geq 1),$及初始值$h(1)=1,h(0)=0$\par 
可得对应生成函数有如下形式:
\begin{align*}
    H(x)&=\sum_{n\geq0}{h(n)x^n}=0+\sum_{n\geq1}{h(n)x^n}=\sum_{n\geq1}{x^n(2h(n-1)+1)}\\
	&=2\sum_{n\geq1}{x^nh(n-1)}+\sum_{n\geq1}{x^n}=2x\sum_{m\geq0}{x^mh(m)}+\frac{1}{1-x}-1\\
	&=2xH(x)+\frac{1}{1-x}-1
\end{align*}\par 
移项可得$ (1-2x)H(x)=\frac{1}{1-x}-1$,进一步整理可得生成函数形式:
\begin{align*}
     H(x)&=\frac{x}{(1-x)(1-2x)}=\frac{1}{1-2x}-\frac{1}{1-x}\\
	&=\sum_{n\geq0}{2^nx^n}-\sum_{n\geq0}{x^n}\\
	&=\sum_{n\geq0}{(2^n-1)x^n}=\sum_{n\geq0}{h(n)x^n}
\end{align*}\par 
系数对比即得$h(n)$的通项公式$h(n)=2^n-1$\par 
事实上,对上述递推式进行推广我们有更一般的情形,如例2.
\begin{example}
\begin{align*}
	\begin{cases}
            h(n)=2h(n-1)+3^n \\
            h(1)=3		
	\end{cases}
\end{align*}
\end{example}
法一:待定系数法\par 
\begin{solution}
    解方程$h(n)+\alpha·3^n=2(h(n-1)+\alpha·3^{n-1})$, 得$\alpha=-3$,\par 
    即得$h(n)-3·3^n=2(h(n-1)-3·3^{n-1})$,\par 
	另设函数$t(n)=h(n)-3·3^n$即有$t(n)=2t(n-1)$及初始值$t(1)=-6$;\par 
	由等比数列易得$t(n)=-3\cdot 2^n$;\par 
	反解即得$h(n)=3(3^n-2^n)$
\end{solution}
~\par 
法二:生成函数\par 
\begin{solution}
	该递推关系对应生成函数有如下形式: 
	\begin{align*}
	H(x)&=\sum_{n\geq0}{h(n)x^n}=0+\sum_{n\geq1}{x^nh(n)}\\
	&=\sum_{n\geq1}{x^n(2h(n-1)+3^n)}\\
	&=2xH(x)+\frac{1}{1-3x}-1	
	\end{align*}\par 
移项可得:$(1-2x)H(x)=\frac{3x}{1-3x}$进一步整理可得生成函数形式:
\begin{equation*}
H(x)=\frac{3x}{(1-3x)(1-2x)}=\frac{3}{1-3x}-\frac{3}{1-2x}
\end{equation*}\par 
系数对比即得$h(n)$的通项公式$ h(n)=3(3^n-2^n)$
\end{solution}
\begin{example}
	\begin{align*}
\begin{cases}
f(n)=f(n-1)+g(n-1)+g(n-2) ,f(0)=f(1)=1\\
g(n)=g(n-1)+2f(n-1)-f(n-2),g(0)=g(1)=1	
\end{cases}		
	\end{align*}
\end{example}
\begin{solution}
考察$f(n),g(n)$对应的生成函数$F(n),G(n)$;
\begin{align*}
F(x)&=\sum_{n\geq0}{f(n)x^n}=1+x+\sum_{n\geq2}{x^nf(n)}\\
&=1+x+\sum_{n\geq2}{x^n(f(n-1)+g(n-1)+g(n-2))}\\
&=1+x+x\sum_{n\geq1}{x^nf(n)}+x\sum_{n\geq1}{x^ng(n)}+x^2\sum_{n\geq0}{x^ng(n)}\\
&=1+x+x(F(x)-1)+x(G(x)-1)+x^2G(x)
\end{align*}
同理可得
\[G(x)=1+x+x(G(x)-1)+2x(F(x)-1)-x^2F(x)\]
整理可得
\begin{align*}
x-1&=(x-1)F(x)+(x+x^2)G(x)\\
2x-1&=(x-1)G(x)+(2x-x^2)F(x)
\end{align*}
解得
\begin{align*}
F(x)&=-\frac{2x^3+x-1}{x^4-x^3-x^2-2x+1}\\
G(x)&=\frac{x^3-x^2-x+1}{x^4-x^3-x^2-2x+1}
\end{align*}\par 
注意到,对于形如$F(x)=\frac{A(x)}{B(x)}$的式子,则可写作部分分式和,从而进行级数展开,$f(n)$即是$x^n$项对应系数.
\end{solution}
\begin{example}
	对于系数为n的递推关系,如:
\begin{align*}
	\begin{cases}
	h(n)=nh(n-1)+3n\\
	h(1)=0		
	\end{cases}
\end{align*}
	\end{example}
	\begin{solution}
	注意到:两边同时除$n!$得
	\[\frac{h(n)}{n!}=\frac{h(n-1)}{(n-1)!}+\frac{3}{(n-1)!}\]
	\[\therefore h(n)=3n!\sum_{i=1}^{n}\frac{1}{(i-1)!}\]\par 
	\end{solution}
	\begin{example}
	对于系数为$n^2$的递推关系,如:
	\begin{align*}
		\begin{cases}
		h(n)=n^2h(n-1)+3n\\
		h(1)=0		
		\end{cases}
	\end{align*}
	\end{example}
	\begin{solution}
	同理有:两边同时除$(n!)^2$得
	\[\frac{h(n)}{(n!)^2}=\frac{h(n-1)}{((n-1)!)^2}+\frac{3n}{(n!)^2}\]
	\[\therefore h(n)=3n(n!)^2\sum_{i=1}^{n}\frac{1}{(i!)^2}\]
	\end{solution}
	一般地,对$h(n)=a_nh(n-1)+b_n$,其中$a_n,b_n$为关于n的多项式,常数或指数等\par 
	则有以下变化形式:
	\begin{equation*}
		\frac{h(n)}{a_n\cdot a_{n-1}\cdots a_1}=\frac{h(n-1)}{a_{n-1}\cdot a_{n-2}\cdots a_1}+\frac{b_n}{a_n\cdot a_{n-1}\cdots a_1}
	\end{equation*}
	\section{指数型生成函数}
	\begin{definition}[指数型生成函数] 
	\end{definition}
	对于数列$a_n$,定义其指数型生成函数为$A(n)=a_0+a_{1}x+a_{2}\frac{x^2}{2!}+\cdots+a_n\frac{x^n}{n!}+\cdots$
	\begin{theorem}[Vandermonde恒等式]\end{theorem}
	第二节课中我们已经介绍过范德蒙德恒等式
    		\begin{equation*}
     		{n+m \choose k}=\sum_{0\le i\le k}{n \choose i}{m \choose k-i}.
     		\end{equation*}\par 
		将其扩展到三阶形式:
		\begin{equation*}
               {n+m+p \choose l}=\sum_{\substack{i,j\neq 0\\i+j\leq l}}{n \choose i}{m \choose j}{p \choose l-i-j}		
		\end{equation*}
		\begin{proof}
            对$(1+x)^n,(1+x)^m,(1+x)^p$使用广义二项式定理
            \begin{equation*}
                (1+x)^n=\sum_{i\ge0}{n \choose i}x^i,\tag{1}
            \end{equation*}
            \begin{equation*}
                (1+x)^m=\sum_{j\ge0}{m \choose j}x^j.\tag{2}
            \end{equation*}
            \begin{equation*}
                (1+x)^p=\sum_{k\ge0}{p \choose k}x^k.\tag{3}
            \end{equation*}\par 
            将$(1),(2),(3)$式相乘得:
            \[(1+x)^{n+m+p} =\sum_{l\ge0}x^{l}\sum_{i,j,k\neq0,i+j+k=l}{n \choose i}{m \choose j}{p \choose k}\notag\]\par 
			由限制条件可消去变量k得:
			\[(1+x)^{n+m+p} =\sum_{l\ge0}x^{l}\sum_{i,j\neq 0,i+j\leq l}{n \choose i}{m \choose j}{p \choose l-i-j}\tag{4}\]\par 
           对$(1+x)^{n+m+p}$使用广义二项式定理可得
           \begin{equation*}
               (1+x)^{n+m+p}=\sum_{l\ge0}x^l{n+m+p \choose l}.\tag{5}
           \end{equation*}\par 
           对比$(4),(5)$式中$x^k$系数,于是得到
           \begin{equation*}
               {n+m+p \choose l}=\sum_{i,j\neq 0,i+j\leq l}{n \choose i}{m \choose j}{p \choose l-i-j}
           \end{equation*}
		\end{proof}
	
	\begin{example}
	有红黄蓝三色的共n个球排成一列,其中红球有偶数个,黄球有奇数个,蓝球不做限制,求共有多少种排列方法?
	\end{example}
	\begin{solution}
	设排列方法共有$f(n)$种,假设红球个数为$i\quad (2\mid i)$,黄球个数为$j\quad (2\nmid j)$,则蓝球个数为$n-i-j\quad (i+j\leq n)$\par 
	由可重排列公式,对确定的i,j有排列数$\frac{n!}{i!j!(n-i-j)!}$;\par 
	故所有可能情形数为\[f(n)=\sum_{\substack{i,j\\ i+j\leq n,2\mid i,2\nmid j}}\frac{n!}{i!j!(n-i-j)!}\]\par 
	假设$f(n)$生成函数为F(n),并设以下三个函数A(n),B(n)和C(n)\par 
	若不对球的奇偶性做限制,有
	\[A(n)=\sum_{i\ge0}\frac{x^i}{i!}\quad B(n)=\sum_{j\ge0}\frac{x^j}{j!}\quad C(n)=\sum_{k\ge0}\frac{x^k}{k!}\]\par 
	考虑$x^n$项系数,则$k$只能取$n-i-j$,由Vandermonde恒等式三阶形式可知:
	\[\sum_{n\geq 0}[\sum_{\substack{i,j\\i+j\leq n}}\frac{1}{i!j!(n-i-j)!}] x^n=\sum_{i\geq 0}\frac{x^i}{i!}\cdot\sum_{j\geq 0}\frac{x^j}{j!}\cdot\sum_{k\geq 0}\frac{x^{k}}{k!}\]\par 
	对红黄两球限制奇偶性$2|i,2\nmid j$可得:
	\[A(n)=1+\frac{x^2}{2!}+\frac{x^4}{4!}+\cdots=\frac{e^x+e^{-x}}{2}\]
	\[B(n)=x+\frac{x^3}{3!}+\frac{x^5}{5!}+\cdots=\frac{e^x-e^{-x}}{2}\]
	\[C(n)=\sum_{i\ge0}\frac{x^i}{i!}=1+x+\frac{x^2}{2!}+\cdots=e^x\]\par 
	考虑红黄两球的奇偶性限制则有:
	\begin{align*}
		\sum_{n\geq 0}\frac{f(n)}{n!}x^n
		&=\sum_{i\geq 0,2\mid i}\frac{x^i}{i!}\cdot\sum_{j\geq 0,2\nmid j}\frac{x^j}{j!}\cdot\sum_{k\geq 0}\frac{x^{k}}{k!}\\
		&=e^x\frac{e^x+e^{-x}}{2}\frac{e^x-e^{-x}}{2}=\frac{1}{4}(e^{3x}-e^{-x})\\
		&=\frac{1}{4}\sum_{n\ge0}(\frac{3^n}{n!}-\frac{(-1)^n}{n!})x^n
	\end{align*}\par 
	对比系数可得:
	\[f(n)=\frac{1}{4}(3^n-(-1)^n)\]\par 
	将其推广到较为复杂的情况,有以下例子。
	\end{solution}
	\begin{example}
		有红黄蓝三色的共n个球排成一列,其中红球有偶数个,黄球个数可被3整除,蓝球不做限制,求共有多少种排列方法?
		\end{example}
	\begin{solution}
		假设排列方法共有$f(n)$种,红球个数为$i\quad (2\mid i)$,黄球个数为$j\quad (3\mid j)$,则蓝球个数为$n-i-j\quad (i+j\leq n)$;
		因此排列方法数为:
		\[f(n)=\sum_{\substack{i,j\\ i+j\leq n,2\mid i,3\mid j}}\frac{n!}{i!j!(n-i-j)!}\]
		考虑红黄球的个数限制,同例6可得下式(其中$w=e^\frac{2\pi i}{3}=-\frac{1}{2}+\frac{\sqrt{3}i}{2}$为三次单位根):
		\begin{align*}
			\sum_{n\geq 0}\frac{f(n)}{n!}x^n
			&=\sum_{i\geq 0,2\mid i}\frac{x^i}{i!}\cdot\sum_{j\geq 0,3\mid j}\frac{x^j}{j!}\cdot\sum_{k\geq 0}\frac{x^{k}}{(k)!}\\
			&=e^x\frac{e^x+e^{-x}}{2}\frac{e^x+e^{wx}+e^{w^2x}}{3}\\
			&=\frac{1}{6}(e^{3x}+e^{(\frac{3+\sqrt{3}i}{2})x}+e^{(\frac{3-\sqrt{3}i}{2})x}+e^x+e^{(\frac{-1+\sqrt{3}i}{2})x}+e^{(\frac{-1-\sqrt{3}i}{2})x})\\
			&=\frac{1}{6}\sum_{n\geq 0}\frac{1}{n!}(3^n+(\frac{3+\sqrt{3}i}{2})^n+(\frac{3-\sqrt{3}i}{2})^n+1+(\frac{-1+\sqrt{3}i}{2})^n+(\frac{-1-\sqrt{3}i}{2})^n)x^n\\
		\end{align*}\par
		对比系数可得:
		\[f(n)=\frac{1}{6}(3^n+(\frac{3+\sqrt{3}i}{2})^n+(\frac{3-\sqrt{3}i}{2})^n+1+(\frac{-1+\sqrt{3}i}{2})^n+(\frac{-1-\sqrt{3}i}{2})^n)\]\par 
		注意到:上式中含有虚数单位$i$的式子有两对共轭复数,考虑到共轭复数的n次幂仍互为共轭复数,其和为实部的两倍,可进一步简化得到:
		\begin{align*}
			f(n)=\frac{1}{6}(3^n+2(\sqrt{3})^n cos(\frac{n\pi}{6})+1+2cos(\frac{2n\pi}{3}))
		\end{align*}
	\end{solution}
	\section{卡特兰数(Catalan Number)}
		回顾汉诺塔问题,要求3根柱子,n个盘子(大小顺序为上小下大),规定堆叠过程中大盘子不能放置于小盘子上方,求将所有盘子保持顺序从最左端移到最右端的方法数。\par 
		卡特兰数初始条件相同,未限制堆叠顺序,但规定了盘子只能单向移动(例如,$I\rightarrow II\rightarrow III$或$I\rightarrow III$),考察最多能在最右端形成多少种不同的排列。\par 
		特别的,卡特兰数在数据结构中有重要应用,栈规定数据先进后出,函数调用时进行递归操作时,不能越过节点直接访问子节点。该问题还可表述为多节车厢通过人字形铁轨能形成的排列数。
	\section{选做题}
		n*n方格图用1*2或2*1的多米诺牌进行覆盖,总共有多少种可能性?请给出显式表达式或估计。
	\end{document}	