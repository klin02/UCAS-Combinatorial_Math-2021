\documentclass[fontset=windows,11pt]{article}
\usepackage[a4paper]{geometry}
\geometry{left=2.0cm,right=2.0cm,top=2.5cm,bottom=2.5cm}

\usepackage[UTF8]{ctex}
\usepackage{amsmath,amsfonts,graphicx,amssymb,bm,amsthm}
\usepackage[noend]{algpseudocode}
\usepackage{mathtools}
\usepackage{graphicx}
\newtheorem{question}{\hskip 1.7em}

\newenvironment{solution}{{\noindent\hskip 2em \bf 解 \quad}}

\renewenvironment{proof}{{\noindent\hskip 2em \bf 证明 \quad}}{\hfill$\qed$\par\vskip1em}
\newcommand\E{\mathbb{E}}

\begin{document}
    \begin{center}
        {\Large \bf 组合数学Homework7}\\
    \end{center}
    \begin{kaishu}
        \hfill 提交者:游昆霖 \quad 学号:2020K8009926006
    \end{kaishu}

    \begin{question}
        证明:对任意集合$S\subset\{1,2,\dots,n\}$,只要$|S|\ge\log_2 n + \log_2 \log_2 n + 1$,就必然存在集
        合$T_1,T_2\subseteq S,T_1\cap T_2=\phi,\sum_{t_1\in T_1}t_1=\sum_{t_2\in T_2}t_2$。
    \end{question}

    \begin{proof}
        记$Sum(T)=\sum_{z\in T}z$,则取$T\subseteq S$,有$0\leq Sum(T)\leq n|S|$\par 
        又因$T\subseteq S$,$T$有$2^{|S|}$种可能;\par 
        $\because f(x)=2^x-nx-1$,令$f'(x)>0$,则$x>log_2n+log_2log_2e$;\par 
        $\therefore$当$x\ge\log_2 n + \log_2 \log_2 n + 1$时,有$f'(x)>0$\par 
        $\therefore 2^{|S|}-n|S|-1\geq nlog_2n-nlog_2log_2n-n-1> 0$(当n$\geq 5$)\par 
        由抽屉原理,$\exists A\neq B\subseteq S,s.t. Sum(A)=Sum(B)$;\par 
        取$T_1=A\backslash(A\cap B),T_2=B\backslash(A\cap B)$,即有$T_1,T_2\subseteq S, T_1\cap T_2=\emptyset, Sum(T_1)=Sum(T_2)$\par 
        当$n=1,2,3,4$时,枚举易证。
    \end{proof}

    \begin{question}
        证明:对任意集合$S\subset\{1,2,\dots,64\}$,$|S|=9$,都必然存
        在集合$T_1,T_2\subset S,T_1\cap T_2 = \emptyset,\sum_{t_1\in T_1}t_1=\sum_{t_2\in T_2}t_2$。
    \end{question}

    \begin{proof}
        $\because |S|=9$,记其中元素从小到大依次为$a_1,...,a_9$,若存在$S$的不同子集$T_1,T_2,s.t.Sum(T_1)=Sum(T_2)$,必有
        $Sum(T_i)\leq a_2+\cdots+a_9$,故对任意可能成立的$T$,有$1\leq Sum(T)\leq 59+\cdots+64=482$,即共482个“抽屉”;\par 
        又因对$S$,满足$T\subseteq S,Sum(T)\leq a_2+\cdots+a_9$的$T$有$2^{|S|}-2=510$种可能;\par 
        由抽屉原理,$\exists A,B\subseteq S,s.t.Sum(A)=Sum(B)$\par 
        取$T_1=A\backslash(A\cap B),T_2=B\backslash(A\cap B)$,即有$T_1,T_2\subseteq S, T_1\cap T_2=\emptyset, Sum(T_1)=Sum(T_2)$
    \end{proof}

    \begin{question}
        某人用$64$天读了$100$页书,其中每天读的页数是一正整数,当$p$取哪些值时,一定存在连续的若干天,在
        这些天此人正好读了$p$页书?对满足条件的$p$给出证明,不满足条件的$p$给出反例。
    \end{question}

    \begin{solution}
        记$S_i$表示前$i$天所看页数,则有$1\leq S_1<S_2<\cdots <S_64=100$,以下对p进行分类讨论:\par 
        (1)当$1\leq p\leq 28$时,若$\exists S_i=p$,成立,否则$\forall i,S_i,S_i+p$均不等于p,且有$1\leq S_1,...,S_{64},S_1+p,...,S_{64}+p\leq 100+p\leq 128$,即“抽屉”小于等于127个,而“苹果”有128个;由抽屉原理
        ,$\exists i,j\quad s.t.S_j=S_i+p$,即有连续若干天读p页书,成立;\par 
        (2)当p=29,...,32时,对$S_p$进行分类讨论:\par 
        $\textcircled{1}$若$S_p\leq 2p-1$,若$\exists 1\leq i\leq p,s.t.S_i\equiv 0$成立,否则$S_1,...S_p$这p个数落在$\{1,p+1\},...,\{p-1,2p-1\}$这p-1个抽屉中,
        由抽屉原理,必有两数落在同一个抽屉中,即相减为p,成立;\par 
        $\textcircled{2}$若$S_p\geq 2p$,定义$t=100-3p$,考虑抽屉$\{2p,3p\},\{2p+1,3p+1\},...,\{2p+t,3p+t\},\{2p+t+1\},...,\{3p-1\}$,共$t+1+3p-1-(2p+t)=p$个抽屉,此时
        此时有$S_p,...,S_{64}$这65-p个数落在p个抽屉里,因为$65-p>p$,由抽屉原理,必有两数落在同一个抽屉中(且不等),即相减为p,成立\par 
        (3)当p=33,...,36时,可举出反例:$S_1=1,...S_{p-1}=p-1,S_p=p+36,...S_{64}=100$\par 
        (4)当$37\leq p\leq 64$时,有$S_p\leq 100-(64-p)=36+p$,若$\exists 1\leq i\leq p,s.t. S_i\equiv 0$,由于$2\times p>36+p$,故$S_i=p$,成立;
        否则模p意义下只有p-1个抽屉,即$\exists i<j, s.t.S_i\equiv S_j(\mod{p})$,由于$2\times p>36+p$,即$S_j-S_i=p$,成立;\par 
        (5)当$p\geq 65$时,只需使得$S_1,...S_{64}$分取模p意义下${1,...,p-1}$的不同数,则任意二者相减不为p,即给出反例。例如当p=65时,取$S_1=36,S_2=37,...,S_{29}=64,S_{30}=66,...,S_{64}=100$.

    \end{solution}

    \begin{question}
        平面上每个点都以红蓝两种颜色染色。求证:存在两个相似比为$2011$的相似三角形,每个三角形的三个顶点同色。
    \end{question}

    \begin{proof}
        以半径1和2011分别做两个同心圆,分别记作$O_1,O_2$,在$O_1$上任取9点,由抽屉原理,必有5点同色,记为$A,B,C,D,E$,过同心圆圆心和对应点的射线分别交$O_2$于
        $A',B',C',D',E'$,则由抽屉原理,其中必有三点同色,不妨记为$A',B',C'$,则有$\triangle ABC \sim \triangle A'B'C'$,相似比为2011,且每个三角形三顶点共色。
    \end{proof}

    \begin{question}
        证明下列命题:
        \begin{itemize}
            \item[a.] $R(3,4) = 9$;
            \item[b.] $R(4,4)\le 18$。
        \end{itemize}
    \end{question}

    \begin{proof}
        a.首先对完全图任一点所引出的边进行讨论:\par 
        (1)若存在一个点引出至少4条红边,考虑对应4个点之间边的着色情形:若存在红边,则存在红色的$K_3$;否则全为蓝边,存在蓝色$K_4$\par 
        (2)若存在一个点引出至少6条蓝边,考虑对应6个点之间边的着色情形:由于$R(3,3)=6$,故6点的边或存在红色$K_3$,或存在蓝色$K_3$,结合原始点,即存在蓝色$K_4$;
        然后考虑每个点引出3条红边,6条蓝边的$K_9$:\par 
        由于此时红边总数为$\frac{3\times 9}{2}$,蓝边总数为$\frac{6\times 9}{2}$,均不为整数,即不存在这样的$K_9$,故对任意$K_9$,必存在某一点,或引出至少4条红边,
        或引出至少6条蓝边,由上述讨论,可得$R(3,4)\leq 9$;
        且对$K_8$,可给出以下既不存在蓝色$K_3$,也不存在红色$K_4$的反例(与图片对应,互换红蓝,由对称性不改变结论):
        \begin{figure}[htbp]
            \centering
            \includegraphics[scale=0.8]{ramsey.png}
        \end{figure}\par 
        b.利用Ramsey递推不等式,$R(n,m)\leq R(n-1,m)+R(n,m-1)$,有$R(4,4)\leq R(3,4)+R(4,3)=18$.
    \end{proof}

    \begin{question}
        证明:给定正整数$d$,存在正整数$N$,使得将$1,2,\dots,N$任意$d$染色时,都存在四个互不相同的正整数$x,y,z,w$,它们颜色相同,且$xy=zw$。
    \end{question}

    \begin{proof}
        首先考虑第一象限平面点阵,其中格点$(x,y)$对应正整数$2^x\cdot 3^y$,取矩形宽为d+1,则边上格点必有一对同色;由于总共d中颜色,且每种颜色的一对同色点
        位置可能性为$\binom{d+1}{2}$,取矩形长为$d\binom{d+1}{2}$,则必存在两列均有一对颜色、位置相同的同色点,将这四个点从左下角顺时针依次记为$A,B,C,D$,
        ,分别对应$x,z,y,w$,且$A,C$对应坐标为$(i,j),(l,k)$,则有$xy=zw=2^{i+l}3^{j+k}$;故存在满足题意的N,取$N\geq 2^l3^k$即可
    \end{proof}

    \begin{question}
        对平面上的整点二染色,证明对任意正整数$m,n$,存在集合$S,T\subset \mathbb{Z}$,满
        足$|S|\geq m,|T|\geq n$,$S$和$T$中的所有元素分别构成两个等差数列,且网格图$S\times T \subseteq \mathbb{Z}^2$中所有格点同色。
    \end{question}

    \begin{proof}
        固定T,观察S,即平行于x轴的一列格点,由范德瓦尔登定理,$\forall m\in \mathbb{Z}$,存在长度为m的同色等差数列,将该数列首尾点中全部点作为一个block,
        记该block长度为$l$,则$l\leq W(2,m)$,且该block染色共有$2^l$种可能;\par 
        改变T,对不同T对应的block,由范德瓦尔登定理,$\forall n\in \mathbb{Z}$,存在长度为n的同色等差数列,则该数列首尾点间全部点数量为k,则$k\leq W(2^l,n)$;\par 
        将同色等差block的纵坐标作为集合T,将block内同色等差数列横坐标作为集合S,则有$S\times T$中所有格点同色。
    \end{proof}

    \begin{question}
        我们用概率方法证明以下结论:存在足够大的正整数$n$,使得当$n$个人两两进行一场比赛(没有平局)时,存在一种比赛结果(指任意两人都已分出胜负),
        使得对任意$4$个人都存在一个人将他们全部打败。下面我们用概率方法证明这个结论,考虑独立等概率随机地选取每场比赛的结果,试回答下列问题:
        \begin{itemize}
            \item[a.] 总共$n$个人互相比赛,任取$4$个人,计算不存在一个人将他们全部打败的概率;
            \item[b.] 证明存在充分大的$n$,此时对任意$4$个人都存在一个人将他们全部打败的概率大于$0$,从而证明该结论。
        \end{itemize}
    \end{question}

    \begin{solution}
        a.取4个人后,考虑剩下的每一个人,无法将4个人全部打败概率为$1-(\frac{1}{2})^4=\frac{15}{16}$,故不存在一个人将4个人全部打败的概率为
        $(\frac{15}{16})^{n-4}$;\par 
        b.定义比赛结果为随机变量$K$,若比赛结果对任意4个人存在一个人将他们全部打败,记为“good”,否则记为“bad”。则有$Pr(K good)=1-Pr(K bad)$\par 
        记对4个人$i_1,i_2,i_3,i_4$,不存在一个人把他们全部打败为事件$A(i_1,i_2,i_3,i_4)$;则由(a)有$A(i_1,i_2,i_3,i_4)=(\frac{15}{16})^{n-4}$:
        故:
        \begin{align*}
            Pr(K bad)&=Pr(\bigcup_{1\leq i_1<\cdots<i_4\leq n}A(i_1,i_2,i_3,i_4))
                &\leq \sum_{1\leq i_1<\cdots<i_4\leq n}Pr(A(i_1,i_2,i_3,i_4))
                &=\binom{n}{4}(\frac{15}{16})^{n-4}
        \end{align*}
        当n充分大,有$Pr(K bad)<1$,则$Pr(K good)>0$,即存在一种比赛结果,任意四个人都存在一个人将其全部打败。

    \end{solution}
\end{document}