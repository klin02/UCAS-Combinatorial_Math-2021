\documentclass[fontset=windows,11pt]{article}
\usepackage[a4paper]{geometry}
\geometry{left=2.0cm,right=2.0cm,top=2.5cm,bottom=2.5cm}

\usepackage[UTF8]{ctex}
\usepackage{amsmath,amsfonts,graphicx,amssymb,bm,amsthm}
\usepackage{algorithm}
\usepackage{algorithmicx}
\usepackage[noend]{algpseudocode}
\usepackage{mathtools}

\newtheorem{question}{\hskip 1.7em}

\newenvironment{solution}{{\noindent\hskip 2em \bf 解 \quad}}

\renewenvironment{proof}{{\noindent\hskip 2em \bf 证明 \quad}}{\hfill$\qed$\par\vskip1em}

\begin{document}
    \begin{center}
        {\Large \bf 组合数学08.30思考题}\\
    \end{center}
    \begin{kaishu}
        \hfill 提交者:游昆霖 \quad 学号:2020K8009926006
    \end{kaishu}
    \begin{question}
        现有 A,B两人分一个蛋糕,A和B对这个蛋糕的不同部分有自己的偏好。试问能否将这个蛋糕
    分成两份,A,B各拿一份,并且要求 A 拿到的蛋糕在他自己的偏好下不少于整个蛋糕的$\frac{\sqrt{5}-1}{2}$,
    B拿到的部分在他自己的偏好下不少于整个蛋糕的$\frac{3-\sqrt{5}}{2}$。


        问题形式化表述如下:现有非负整数$f_1.f_2$满足
        \[\int_0^1f_i(x)dx=1(i=1,2)\]
        是否存在区间[0,1]的子集A,B使得$A\bigcup B=[0,1],A\bigcap B=\varnothing$且\\
        \[ \int_Af_1(x)dx \geq \frac{\sqrt{5}-1}{2}, \quad \int_Bf_2(x)dx \geq \frac{3-\sqrt{5}}{2} ?\]
        试为这两人的要求设计一个算法,或证明划分不存在。
    \end{question}
    \begin{solution}
       
        \begin{center}
        首先,注意到:黄金分割率为斐波那契数列递推关系的特征根,具体如下:\\   
        设斐波那契数列第n项为$F_n$,则 $F_1=F_2=1,F_{n+1}=F_n+F_{n-1}$;\\
        由特征根方程可得:$F_n=\alpha(\frac{1+\sqrt{5}}{2})^n+\beta(\frac{1-\sqrt{5}}{2})^n$\\
        结合$F_1,F_2$可确定$\alpha=\frac{1}{\sqrt{5}},\beta=-\frac{1}{\sqrt{5}}$;\\
        当$n\gg 1$时,$\beta$项衰减震荡消失,即可得$\frac{F_n}{F_{n+1}}\rightarrow \frac{\sqrt{5}-1}{2}$;\\
        根据以上性质,可确定分蛋糕算法如下:
        \end{center}
    \end{solution}
    \begin{algorithm}
        \caption{Divide cake by golden ratio}
    \textbf{输入}:$f_1(x) f_2(x) \backslash\backslash$ A和B对应的蛋糕函数 \\
    \textbf{输出}:n 满足黄金分割比的分割蛋糕方法\\
    \textbf{function} Divide( )
  
    \qquad $F_1\leftarrow 1,F_2\leftarrow 1,n\leftarrow 1$;

    \qquad \textbf{while} true \textbf{do}
 
    \qquad\qquad Let $0=x_1<x_2<\cdots <x_{F_{n+1}+1}=1,$s.t.$\int_{x_1}^{x_2}f_1(x)dx=\cdots=\int_{x_{F_{n+1}}}^{x_{F_{n+1}+1}}f_1(x)dx=\frac{1}{F_{n+1}}$ 

    \qquad\qquad $\exists 0\leq x_{i_1}<\cdots<x_{i_{F_n+1}}\leq 1,B=\bigcup_{k=1}^{F_n}[x_{i_k},x_{i_{k+1}}],$s.t. $\int_Bf_2(x)$取得最大值;
    
    \qquad\qquad $\backslash \backslash$ A将蛋糕按自己价值标准平均分割为$F_{n+1}$块;B取其中对自己价值较大的$F_n$块;

    \qquad\qquad \textbf{if} $\int_Bf_2(x) \geq \frac{3-\sqrt{5}}{2}$ \textbf{then} break;$\backslash\backslash$ 若B所取蛋糕满足条件,跳出循环

    \qquad\qquad \textbf{else} $n\leftarrow n+1, F_{n+1}\leftarrow F_n+F_{n-1}$
        
    \qquad\qquad \textbf{end if}

    \qquad\textbf{end while}

    \qquad\textbf{return} n \\
    \textbf{end function}
    \end{algorithm}
    注:由上述斐波那契前后两项比值趋于黄金分割比可知,该算法进行下去,最终能求得满足题意的解法。    
\end{document}