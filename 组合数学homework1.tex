\documentclass[fontset=windows,11pt]{article}
\usepackage[a4paper]{geometry}
\geometry{left=2.0cm,right=2.0cm,top=2.5cm,bottom=2.5cm}

\usepackage[UTF8]{ctex}
\usepackage{amsmath,amsfonts,graphicx,amssymb,bm,amsthm}
\usepackage[noend]{algpseudocode}
\usepackage{mathtools}
\newtheorem{question}{\hskip 1.7em}

\newenvironment{solution}{{\noindent\hskip 2em \bf 解 \quad}}

\renewenvironment{proof}{{\noindent\hskip 2em \bf 证明 \quad}}{\hfill$\qed$\par\vskip1em}
\newcommand\E{\mathbb{E}}

\begin{document}
    \begin{center}
        {\Large \bf 组合数学Homework1}\\
    \end{center}
    \begin{kaishu}
        \hfill 提交者:游昆霖 \quad 学号:2020K8009926006
    \end{kaishu}

    \begin{question}
        计算$\binom{-2}{m}$,并验证它是$(1+x)^{-2}$展开式的第$m$项的系数。
    \end{question}

    \begin{solution}
        \begin{gather*}
           \binom{-2}{m}=\frac{(-2)^{\underline{m}}}{m!}
           =\frac{(-2)(-2-1)\cdots(-2-m+1)}{m!}
           =\frac{(-1)^m(m+1)!}{m!}
           =(-1)^m(m+1) \\
           \text{由麦克劳林展开式} \quad f(x)=f(0)+f'(0)x+\frac{f''(0)}{2!}x^2+\cdots
           +\frac{f^{(n)}}{n!}x^n+\cdots \\
           \text{故}(1+x)^{-2} \text{展开式第$m$项系数为}\\
           \frac{[(1+x)^{-2}]^{(m)}}{m!}
           =\frac{(-2)(-2-1)\cdots(-2-m+1)}{m!}
           =\frac{(-1)^m(m+1)!}{m!}
           =(-1)^m(m+1) \\
           \text{与$\binom{-2}{m}$相等,证毕。}
        \end{gather*}
    \end{solution}

    \begin{question}
        证明:$(1+x)^{\alpha}=\sum\nolimits_{i\geq0}\binom{\alpha}{i}x^i$,
        其中$0<|x|<1$且$\alpha\in\mathbb{R}.$
    \end{question}

    \begin{proof}
        \begin{gather*}
            \text{由麦克劳林展开式} \quad f(x)=f(0)+f'(0)x+\frac{f''(0)}{2!}x^2+\cdots
            +\frac{f^{(n)}}{n!}x^n+\cdots \\
            \text{故}(1+x)^{\alpha} \text{展开式第$i(i>0)$项系数为}\\
            \frac{{(1+x)^{\alpha}}^{(i)}}{i!}
            =\frac{\alpha(\alpha-1)\cdots(\alpha-i+1)}{i!}
            =\frac{\alpha^{\underline{i}}}{i!}=\binom{\alpha}{i}\\
        \end{gather*}
        \begin{center}
        展开式首项系数$1=\binom{\alpha}{0}$\\
        展开式与求和式x各次方对应系数相等,故两式相等。            
        \end{center}
    \end{proof}

    \begin{question}
        计算下列和式:\\
        \indent a.$\sum\nolimits_{k\geq0}\binom{n}{3k},\sum\nolimits_{k\geq0}\binom{n}{3k+1},\sum\nolimits_{k\geq0}\binom{n}{3k+2};$\\
        \indent b.$\sum\nolimits_{k=0}^nk^2\binom{n}{k};$\\
        \indent c.$\sum\nolimits_{i=0}^n\frac{1}{i+1}\cdot\binom{n}{i};$\\
        \indent d.$\sum\nolimits_{i=0}^ni^3.$\\    
    \end{question}

    \begin{solution}(a)
        \begin{gather*}
        \because w=e^{i\frac{2\pi}{3}} \text{为三次单位根}\\
        \therefore 1+w+w^2=0\\
        \text{由第2题结论}(1+x)^{\alpha}=\sum\nolimits_{i\geq0}\binom{\alpha}{i}x^i 
        \text{分别取$x=1,w,w^2$即有}\\
        (1)\quad (1+1)^n=\binom{n}{0}+\binom{n}{1}+\binom{n}{2}+\binom{n}{3}+\cdots;\\
        (2)\quad (1+w)^n=\binom{n}{0}+\binom{n}{1}w+\binom{n}{2}w^2+\binom{n}{3}+\cdots;\\
        (3)\quad (1+w^2)^n=\binom{n}{0}+\binom{n}{1}w^2+\binom{n}{2}w+\binom{n}{3}+\cdots;\\
        \therefore 
        \sum\nolimits_{k\geq0}\binom{n}{3k}=\frac{1}{3}[(1+1)^n+(1+w)^n+(1+w^2)^n]\\
        =\frac{1}{3}[2^n+e^{i\frac{n\pi}{3}}+e^{i\frac{-n\pi}{3}}]\\
        =\begin{cases}
        \frac{1}{3}(2^n+2) & \text{if } n=6k,\\
        \frac{1}{3}(2^n+1) & \text{if } n=6k+1 \text{ or } 6k+5,\\ 
        \frac{1}{3}(2^n-1) & \text{if } n=6k+2 \text{ or } 6k+4,\\ 
        \frac{1}{3}(2^n-2) & \text{if } n=6k+3.\\ 
        \end{cases} (k\in\mathbb{R})\\
        \sum\nolimits_{k\geq0}\binom{n}{3k+1}=\frac{1}{3}[(1+1)^n+w^2(1+w)^n+w(1+w^2)^n]\\
        =\frac{1}{3}[2^n+e^{i\frac{(n-2)\pi}{3}}+e^{i\frac{-(n-2)\pi}{3}}]\\
        =\begin{cases}
        \frac{1}{3}(2^n+2) & \text{if } n=6k+2,\\
        \frac{1}{3}(2^n+1) & \text{if } n=6k+1 \text{ or } 6k+3,\\ 
        \frac{1}{3}(2^n-1) & \text{if } n=6k \text{ or } 6k+4,\\ 
        \frac{1}{3}(2^n-2) & \text{if } n=6k+5.\\ 
        \end{cases} (k\in\mathbb{R})\\
        \sum\nolimits_{k\geq0}\binom{n}{3k+2}=\frac{1}{3}[(1+1)^n+w(1+w)^n+w^2(1+w^2)^n]\\
        =\frac{1}{3}[2^n+e^{i\frac{(n+2)\pi}{3}}+e^{i\frac{-(n+2)\pi}{3}}]\\
        =\begin{cases}
        \frac{1}{3}(2^n+2) & \text{if } n=6k+4,\\
        \frac{1}{3}(2^n+1) & \text{if } n=6k+3 \text{ or } 6k+5,\\ 
        \frac{1}{3}(2^n-1) & \text{if } n=6k \text{ or } 6k+2,\\ 
        \frac{1}{3}(2^n-2) & \text{if } n=6k+1.\\ 
        \end{cases} (k\in\mathbb{R})\\        
    \end{gather*}
    \end{solution}

    \begin{solution}(b)(解法一)
        \begin{gather*}
            \text{首先,由公式}k\binom{n}{k}=n\binom{n-1}{k-1} \text{可得:}
            \sum\nolimits_{k=0}^nk\binom{n}{k}=n\sum\nolimits_{k=0}^{n-1}\binom{n-1}{k}=n\cdot 2^{n-1};\tag*{(1)}\\
            \text{另一方面,}k(k-1)\binom{n}{k}=n(k-1)\binom{n-1}{k-1}=n(n-1)\binom{n-2}{k-2};\\
            \text{故}\sum\nolimits_{k=0}^nk(k-1)\binom{n}{k}=\sum\nolimits_{k=2}^nn(n-1)\binom{n-2}{k-2}
            =n(n-1)\sum\nolimits_{k=0}^{n-2}\binom{n-2}{k}=n(n-1)2^{n-2};\tag*{(2)}\\
            \text{将式(1)、(2)相加即得:}
            \sum\nolimits_{k=0}^nk^2\binom{n}{k}=n(n+1)2^{n-2}
        \end{gather*}
    \end{solution}

    \begin{solution}(b)(解法二)
        \begin{gather*}
            \text{对}(1+x)^n=\sum\nolimits_{k=0}^n\binom{n}{k}x^k \text{两侧求一阶、二阶导可得:}\\
            n(1+x)^{n-1}=\sum\nolimits_{k=0}^nk\binom{n}{k}x^{k-1} \tag*{(1)}\\
            n(n-1)(1+x)^{n-2}=\sum\nolimits_{k=0}^nk(k-1)\binom{n}{k}x^{k-2} \tag*{(2)}\\
            \text{式(1)(2)取x=1并相加即得:}\sum\nolimits_{k=0}^nk^2\binom{n}{k}=n(n+1)2^{n-2}\\
        \end{gather*}
    \end{solution}

    \begin{solution}(c)(解法一)
        \begin{gather*}
            \text{由公式} \frac{1}{i+1}\binom{n}{i}=\frac{1}{n+1}\binom{n+1}{i+1}\text{可得:} \\
            \sum\nolimits_{i=0}^n\frac{1}{i+1}\binom{n}{i}=\frac{1}{n+1}\sum\nolimits_{i=0}^n\binom{n+1}{i+1}
            =\frac{1}{n+1}\sum\nolimits_{i=1}^{n+1}\binom{n+1}{i}
            =\frac{1}{n+1}(2^{n+1}-1)
        \end{gather*}
    \end{solution}

    \begin{solution}(c)(解法二)
        \begin{gather*}
            \text{对}(1+x)^n=\sum\nolimits_{k=0}^n\binom{n}{i}x^i \text{两侧求积分可得:}\\
            \frac{1}{1+x}^{n+1}=\sum\nolimits_{k=0}^n\frac{1}{i+1}\binom{n}{i}x^{i+1}\\
            \text{取上式两侧从0到1的定积分即得:}\sum\nolimits_{i=0}^n\frac{1}{i+1}\cdot\binom{n}{i} =\frac{1}{n+1}(2^{n+1}-1)
        \end{gather*}
    \end{solution}

    \begin{solution}(d)(解法一)
        \begin{gather*}
        \text{先证平方和公式,由} (i+1)^3-i^3=3i^2+3i+1 \text{,对i=1,2,$\cdots$,n各式累加并整理,即得:}\\
        \sum\nolimits_{i=0}^ni^2=\frac{1}{3}[(n+1)^3-1]-\frac{n(n+1)}{2}-\frac{n}{3}=\frac{n(n+1)(2n+1)}{6};\\ 
        \text{再由} (i+1)^4-i^4=4i^3+6i^2+4i+1 \text{,对i=1,2,$\cdots$,n各式累加并整理,即得:}\\
        \sum\nolimits_{i=0}^ni^3=\frac{1}{4}[(n+1)^4-1]-\frac{3}{2}\sum\nolimits_{i=0}^ni^2-\frac{n(n+1)}{2}-\frac{n}{4}
        =[\frac{n(n+1)}{2}]^2
        \end{gather*}
    \end{solution}

    \begin{solution}(d)(解法二)
        \begin{gather*}
            \because i^3=i(i-1)(i-2)+3m(m-1)+m \\
            \therefore \sum\nolimits_{i=0}^ni^3=6\sum\nolimits_{i=0}^n\binom{m}{3}+6\sum\nolimits_{i=0}^n\binom{m}{2}+\sum\nolimits_{i=0}^n\binom{m}{1};\\
            \text{由朱世杰恒等式可知:}\sum\nolimits_{i=0}^ni^3=6\binom{n+1}{4}+6\binom{n+1}{3}+\binom{n+1}{2}=[\frac{n(n+1)}{2}]^2\\
        \end{gather*}
    \end{solution}

    \begin{question}
        证明:对任意$k\in \mathbb{N} ,x^k$可以表示为{$x^{\underline{k}},x^{\underline{k-1}},\cdots,x$}的线性组合。
    \end{question}

    \begin{proof}
        \begin{center}
        通过强归纳法证明:显然,当$k=0,1$时成立;\\
        假设对$1\leq k\leq m$成立,即 $x^m,x^{m-1},\cdots x$均可表示为$x^{\underline{m}},x^{\underline{m-1}},\cdots x$的线性组合;\\
        当$x=m+1$时,$x^{m+1}-x^{\underline{m+1}}$为关于x的m次整系数多项式,\\
        由假设,多项式中各项均可表示为$x^{\underline{m}},x^{\underline{m-1}},\cdots x$的线性组合;\\
        则$x^{m+1}$可表示为$x^{\underline{m+1}},x^{\underline{m}},\cdots x$的线性组合;\\
        由强归纳法可知:对任意$k\in \mathbb{N} ,x^k$可以表示为{$x^{\underline{k}},x^{\underline{k-1}},\cdots,x$}的线性组合。    
        \end{center}
    \end{proof}

    \begin{question}
        证明:$(\frac{n}{k})^k \leq \binom{n}{k} \leq (\frac{en}{k})^k$,其中,$k$为正整数且$k\leq n$。
    \end{question}

    \begin{proof}
        \begin{gather*}
            \text{先证左边:} \because \binom{n}{k}=\frac{n}{k}\cdot\frac{n-1}{k-1}\cdots\frac{n-k+1}{1};\\
            \text{对其中第i项均有}\frac{n-i+1}{k-i+1}\geq\frac{n}{k};\\
            \text{将各式累乘即得左式;}\\
            \text{再证右边:}\binom{n}{k} \leq \sum\nolimits_{i=0}^k\binom{n}{i}\\
            \sum\nolimits_{i=0}^k\binom{n}{i}(\frac{k}{n})^k \leq \sum\nolimits_{i=0}^k\binom{n}{i}(\frac{k}{n})^i
            \leq \sum\nolimits_{i=0}^k\binom{n}{i}(\frac{k}{n})^i =[(1+\frac{k}{n})^{\frac{n}{k}}]^k\leq e^k\\
            \text{整理即得:} \binom{n}{k} \leq (\frac{en}{k})^k
        \end{gather*}
    \end{proof}

    \begin{question}
        4名男士和8名女士围着一张圆桌就座,如果每两名男士之间恰好有两名女士,一共有多少种就座方法?
    \end{question}
    
    \begin{solution}
       先考虑男士座位,圆排列共有$3!=6$种可能,每种情形对应的女士座位情况有$8!=40320$种可能,故总共有$3!8!=241920$种可能。
    \end{solution}

    \begin{question}
        15个人围着一张圆桌就座,如果B拒绝坐在A旁边,一共有多少种就座方法?如果B只拒绝坐在A右边一共有多少种就座方法?
    \end{question}

    \begin{solution}
        首先考虑A和B的位置,有12种可能,每种情形对应的其他人就座情况有$13!=6227020800$种可能,故总共有$12\cdot 13!=74724249600$种可能。
    \end{solution}

    \begin{question}
        在一个聚会上有15名男士和20名女士,请问有多少种方式形成10对男女共舞?
    \end{question}

    \begin{solution}
        首先考虑参加舞蹈的男士,共有$\binom{15}{10}=3003$种可能,下面考虑每种情形女士情况:\\
        考虑男士依次从女士中挑选舞伴,共有$P(20,10)=20^{\underline{10}}=670442572800$种可能;\\
        故总共有$\binom{15}{10}\binom{20}{10}10!$种可能
    \end{solution}

    \begin{question}
        确定下面的多重集合的10排列的数目(10排列指包含10个元素的排列)。
        \[
            S=\{3\cdot a,4\cdot b,5\cdot c\}=\{a,a,a,b,b,b,b,c,c,c,c,c\}
        \]
    \end{question}

    \begin{solution}
        S的10排列可以被划分为6个部分:
        \begin{gather*}
            (1) \{1\cdot a,4\cdot b,5\cdot c\} \text{的10排列数,有} \frac{10!}{1!4!5!}=1260\\
            (2) \{3\cdot a,2\cdot b,5\cdot c\} \text{的10排列数,有} \frac{10!}{3!2!5!}=2520\\
            (3) \{3\cdot a,4\cdot b,3\cdot c\} \text{的10排列数,有} \frac{10!}{3!4!3!}=4200\\
            (4) \{2\cdot a,3\cdot b,5\cdot c\} \text{的10排列数,有} \frac{10!}{2!3!5!}=2520\\
            (5) \{2\cdot a,4\cdot b,4\cdot c\} \text{的10排列数,有} \frac{10!}{2!4!4!}=3150\\
            (6) \{3\cdot a,3\cdot b,4\cdot c\} \text{的10排列数,有} \frac{10!}{3!4!4!}=1050\\
            \text{故S的10排列个数是 }1260+2520+4200+2520+3150+1050=14690 
        \end{gather*}
    \end{solution}

    \begin{question}
        考虑大小为2n的多重集合{$n\cdot a,1,2,3,\cdots,n$},确定它的n组合数。\\
        注:对于多重集合$S$,其n组合是$S$中的n个对象的无序选择。例如对于多重集合$S=\{2\cdot a,b,3\cdot c\}$,
        其3组合是$\{2\cdot a,b\},\{2\cdot a,c\},\{a,b,c\},\{a,2\cdot c\},\{b,2\cdot c\},\{3\cdot c\}$。
    \end{question}

    \begin{solution}
        \begin{center}
        $\because$多重集合中有n个相同元素$a$,其n组合可划分为n+1个部分,\\
        分别对应$a$个数为$0,1,\cdots,n$的情况;\\
        $\therefore$ n组合数为$\sum\nolimits_{k=0}^n\binom{n}{k}=2^n$     
        \end{center}
    \end{solution}

\end{document}