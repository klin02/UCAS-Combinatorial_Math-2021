\documentclass[fontset=windows,11pt]{article}
\usepackage[a4paper]{geometry}
\geometry{left=2.0cm,right=2.0cm,top=2.5cm,bottom=2.5cm}

\usepackage[UTF8]{ctex}
\usepackage{amsmath,amsfonts,graphicx,amssymb,bm,amsthm}
\usepackage[noend]{algpseudocode}
\usepackage{mathtools}
\newtheorem{question}{\hskip 1.7em}

\newenvironment{solution}{{\noindent\hskip 2em \bf 解 \quad}}

\renewenvironment{proof}{{\noindent\hskip 2em \bf 证明 \quad}}{\hfill$\qed$\par\vskip1em}
\newcommand\E{\mathbb{E}}

\begin{document}
    \begin{center}
        {\Large \bf 组合数学Homework3}\\
    \end{center}
    \begin{kaishu}
        \hfill 提交者:游昆霖 \quad 学号:2020K8009926006
    \end{kaishu}

\begin{question}
    求解如下递推关系:
    \begin{itemize}
        \item [a.]$h_{0}=0;\quad h_{1}=1;\quad h_{2}=2;$
            \par\hskip12pt$h_{n}=h_{n-1}+9h_{n-2}-9h_{n-3}, \qquad (n\geq 3)$
        \item [b.]$h_{0}=2;$
            \par\hskip12pt$h_{n}=(n+2)h_{n-1}+(n+2), \qquad (n\geq 1)$
        \item [c.]$h_0=3;\quad h_1=4;$
            \par\hskip12pt$h_n=2h_{n-1}-h_{n-2}+n+1, \qquad (n\ge 2)$
    \end{itemize}
\end{question}

\begin{solution}
    a.(法一)由递推式可得其特征根方程为:$x^3-x^2-9x+9=0$,解得,$x_0=1,x_1=3,x_2=-3$;\par 
    于是$h_n$通项公式形如:$h_n=a+b\times 3^n+c\times (-3)^n$;\quad (a,b,c为常数)\par 
    代入初始值   $h_{0}=0;\quad h_{1}=1;\quad h_{2}=2;$ 可得$a=-\frac{1}{4},b=\frac{1}{3},c=-\frac{1}{12}$;\par 
    于是有通项公式$h_n=-\frac{1}{4}+3^{n-1}+\frac{1}{4}(-3)^{n-1}$\par 
    \qquad(法二)递推式整理可得:$h_n-h_{n-1}=9(h_{n-2}-h_{n-3})$,设$f_n=h_n-h_{n-1}$,\par 
    则有$h_n=\sum_{k=1}^nf_k$,且有初始值$f_1=1,f_2=1$及递推式$f_n=9f_{n-2} \quad(n\geq 3)$故$f_n=9^{\lceil \frac{n}{2}-1\rceil}$;\par 
    \begin{align*}
        \therefore h_n&=\begin{cases}
            2(\sum_{k=0}^{\frac{n}{2}-1}9^k)&=\frac{3^n-1}{4}  \qquad \qquad \text{n为偶数}\\
            2(\sum_{k=0}^{\frac{n-1}{2}-1}9^k)+9^{\frac{n-1}{2}} &=\frac{5}{4}\times 3^{n-1}-\frac{1}{4} \quad \text{n为奇数}
        \end{cases}\\ 
            &=-\frac{1}{4}+3^{n-1}+\frac{1}{4}(-3)^{n-1}
    \end{align*}

    \qquad b.将递推关系式等号两侧同除$\frac{1}{(n+2)!}$,并设$f_n=\frac{h_n}{(n+2)!}$可得递推式$f_n=f_{n-1}+\frac{1}{(n+1)!}$\par 
    由于所得递推式为等差数列,结合初始值$f_0=1$即得$f_n=1+\frac{1}{2!}+\cdots+\frac{1}{(n+1)!}$,\par 
    故$h_n=(n+2)!\times \sum_{k=1}^{n+1}\frac{1}{k!}$\par 
    \qquad c.递推式整理可得:$h_n-h_{n-1}=h_{n-1}-h_{n-2}+n+1$,设$f_n=h_n-h_{n-1}$\par 
    则有$h_n=\sum_{k=1}^nf_k+3$,且有初始值$f_1=1$及递推式$f_n=f_{n-1}+n+1$,\par 
    故$f_n=1+3+4+\cdots+(n+1)=\frac{(n+1)(n+2)}{2}-2=\frac{n^2+3n-2}{2}$\par 
    于是$h_n=\frac{n(n+1)(2n+1)}{12}+\frac{3n(n+1)}{4}-n+3=\frac{n^3+6n^2-n}{6}+3$
\end{solution}

\begin{question}
    设$S$是多重集合$\{\infty\cdot e_{1},\infty\cdot e_{2},\infty \cdot e_{3},\infty\cdot e_{4}\}$。请确定数列$h_{1},h_{2},\dots,h_{n},\dots$的生成函数,其中$h_{n}$表示分别满足下面各种限制的$S$的$n$组合数:
    \begin{itemize}
        \item[a.] 每个$e_{i}$出现的次数为3的倍数;
        \item[b.] $e_{1}$不出现,$e_{2}$至多出现一次;
        \item[c.] 每个$e_{i}$至少出现10次。
    \end{itemize}
\end{question}

\begin{solution}
    设$a_n$表示单重集合$\{\infty\cdot e_{1}\}$的n组合数,其生成函数为$A(x)$,类似的定义$B(x),C(x),D(x)$,则$h_n$的生成函数$H(x)$为满足条件的$A(x),B(x),C(x),D(x)$的乘积\par 
    a.由条件限制可得:$A(x)=B(x)=C(x)=D(x)=1+x^3+x^6+\cdots +x^{3k}+\cdots$\par 
    \[\therefore H(x)=(\sum_{k=0}^\infty x^{3k})^4=(\frac{1}{1-x^3})^4\]\par 
    b.由条件限制可得:$A(x)=1,B(x)=1+x,C(x)=D(x)=1+x+x^2+\cdots +x^k+\cdots$\par 
    \[\therefore H(x)=(1+x)(\sum_{k=0}^\infty x^k)^2=\frac{1+x}{(1-x)^2}\]\par 
    c.由条件限制可得:$A(x)=B(x)=C(x)=D(x)=x^{10}+x^{11}+\cdots +x^k+\cdots$\par 
    \[\therefore H(x)=(\sum_{k=10}^\infty x^k)^4=(\frac{x^{10}}{1-x})4\]
\end{solution}

\begin{question}
    确定满足下面条件,给$1\times n$的棋盘染色的方案数:用红色、蓝色、绿色和橙色着色,要求其中红色格子有奇数个,绿色格子必定出现且有偶数个。
\end{question}

\begin{solution}
    假设方案数共有$f_n$种,假设红格个数i,绿格个数j,蓝格个数k,则橙格个数为$n-i-j-k(i+j+k\leq n)$;
    由可重排列公式,所有可能数为
    \[f_n=\sum_{\substack{i,j,k\\ i+j+k\leq n, 2\nmid i,2\mid j}} \frac{n!}{i!j!k!(n-i-j-k)!}\]\par 
    设红格、绿格、蓝格、橙格生成函数分别为$A(x),B(x),C(x),D(x)$,由奇偶性限制则有:
    \begin{align*}
        A(x)=x+\frac{x^3}{3!}+\frac{x^5}{5!}+\cdots=\frac{e^x-e^{-x}}{2}\\
        B(x)=\frac{x^2}{2!}+\frac{x^4}{4!}+\cdots=\frac{e^x+e^{-x}}{2}-1\\
        C(x)=\sum_{k\ge0}\frac{x^i}{i!}=1+x+\frac{x^2}{2!}+\cdots=e^x\\
        D(x)=\sum_{t\ge0}\frac{x^i}{i!}=1+x+\frac{x^2}{2!}+\cdots=e^x
    \end{align*} \par 
    考虑$x^n$项系数,则t只能取$n-i-j-k$,由Vandermonde恒等式四阶形式可得:
    \begin{align*}
        \sum_{n\geq 0}\frac{f(n)}{n!}x^n&=A(x)\cdot B(x)\cdot C(x)\cdot D(x)\\
        &=\frac{e^x-e^{-x}}{2}\cdot (\frac{e^x+e^{-x}}{2}-1)\cdot e^x \cdot e^x\\
        &=\frac{1}{4}(e^{4x}-2e^{3x}+2e^{x}-1)  \\
        &=-\frac{1}{4}+\frac{1}{4}\sum_{n\geq 0}(\frac{4^n-2\cdot 3^n+2}{n!})x^n  
\end{align*}\par 
    对比系数可得$f_n=\frac{1}{4}(4^n-2\cdot 3^n+2)$
\end{solution}

\begin{question}
    给定一个$n$个节点的扇形(记为$W_{n}$,见下图),问$W_{n}$有多少棵不同的生成树?(图中顶点各不相同)
\end{question}

\begin{solution}
    假设n节点扇形的生成树个数为$h_n$,则显然有$h_1=1,h_2=1,h_3=3$,以下考虑n+1个节点扇形生成树数量$h_{n+1}$:\par
    考虑扇形最右侧节点(记为x)在生成树中的度数,显然其度数只能为1(对应叶)或者2(对应内点),分情形进行考虑;\par 
    a.若该点度数为1,则该点为叶,每种n节点扇形的生成树将最右侧节点(记为a)或扇心节点(记为b)作为x的父节点均可得到n+1节点,且x度数为1的生成树,
    该类情形有$2h_n\quad (n\geq 2)$种\par 
    b.若该点度数为2,则该点为内点,且同时连接n节点扇形最右侧点(a)和扇心(b),将生成树删去点x即可得到含扇心k节点扇形$1\leq k\leq n$的生成树和n-k节点弧形的生成树;
    注意到弧形的生成树唯一,于是该类情形生成树为$h_1+h_2+\cdots +h_{n-1}$种\par 
    综上,有递推式$h_{n+1}=2h_n+\sum_{k=0}^{n-1}h_k\quad (n\geq 2)$;\par 
    整理即得$h_{n+2}=3h_{n+1}-h_n$,结合初始值$h_2=1,h_3=3$解得$h_n=\frac{1}{\sqrt{5}}[(\frac{3+\sqrt{5}}{2})^{n-1}-(\frac{3-\sqrt{5}}{2})^{n-1}]$
\end{solution}

\begin{question}
    计算如下数列:
    \begin{itemize}
        \item [a.]有$n$个叶子节点的不同的满二叉树(full binary tree, 指每个节点孩子个数均为$0$或$2$的树)共有多少种?
        \item [b.]长为$2n$的由左右小括号组成的序列中,包含$n$对可以合法匹配的括号的序列共有多少种?
        \item [c.]给定一个任意的凸$n$边形,不相交地连接多边形的顶点,可以将原先的多边形划分成三角形的组合,问一共有多少种不同的划分方法?
    \end{itemize}
\end{question}

\begin{solution}
    a.假设n个叶子节点的不同的满二叉树共有$h_n$种,显然有$h_1=1,h_2=1$;\par 
    当$n\geq 2$时,根节点必有两个子节点,则有左子树和右子树,且子树叶子均非空,不妨设左子树叶子数为$k\quad (1\leq k \leq n-1)$,则右子树叶子数为$n-k$;\par 
    于是有递推式$h_n=\sum_{k=1}^{n-1}h_k\cdot h_{n-k}$\par 
    补充定义$h_0=1$,设$h_n$的生成函数为$H(x)$,则:
    \begin{align*}
    H(x)&=\sum_{n\geq 0}h_nx^n=1+x+\sum_{n\geq 2}x^n(\sum_{k=1}^{n-1}h_k\cdot h_{n-k})  \\
    \because H^2(x)&=1+2x+\sum_{n\geq 2}x^n(2h_n+\sum_{k=1}^{n-1}h_k\cdot h_{n-k})\\
    &=1+2x+3(H(x)-1-x)
    \end{align*}  \par 
    整理得:$H^2(x)-3H(x)+x+2=0$,解得$H(x)=\frac{3-\sqrt{1-4x}}{2}$(代入$H(0)=h_0=1$可将另解舍去),于是有:\par 
    \begin{align*}
        H(x)&=\frac{1}{2}[2-\sum_{n>=1}x^n\frac{(-1)^{n-1}}{2^{2n-1}}\frac{\binom{2n-2}{n-1}}{n}(-4)^n]\\
        &=1+\sum_{n\geq 1}x^n\frac{\binom{2n-2}{n-1}}{n}
    \end{align*}
    于是$h_n=\frac{\binom{2n-2}{n-1}}{n}\quad (n\geq 1)$\par 
    ~\par 
    b.首先考虑所有可能序列,即在2n位置中选择n个放置左括号,情形数为$\binom{2n}{n}$;\par 
    然后考虑非法输入,即某位置之前右括号数大于左括号数的情形,不妨将问题转化为以下形式:\par 
    考虑$n*n$网格,以纵坐标y表示现有左括号数,横坐标x表示现有右括号数,则每种可能序列均对应从$(0,0)$到$(n,n)$的一个最短路径;且其中非法路径必与
    直线$l:y=x-1$有公共点,做$(0,0)$关于l的对称点$(1,-1)$,则非法序列数等于从$(1,-1)$到$(n,n)$的最短路径数,即$\binom{2n}{n+1}$;\par 
    于是合法序列数为$\binom{2n}{n}-\binom{2n}{n+1}=\frac{1}{n+1}\binom{2n}{n}$\par 
    ~\par 
    c.首先将凸n边形的顶点依次标记为$a_1,a_2,\cdots, a_n$,由于凸多边形任意一条边必属于划分中的一个三角形,不妨取边$(a_1,a_n)$,再在剩余顶点中任取一个不同于以上两点的
    顶点$a_k$构成三角形,则该三角形将凸n边形划分为两个较小的凸多边形,一个是由$a_1,a_2,\cdots,a_k$构成的凸k边形,一个是由$a_k,a_{k+1},\cdots,a_n$构成的凸n-k+1边形;\par 
    设凸n边形对应的三角形划分方法数为$h_n$,显然有$h_3=1$,补充定义$h_2=1$,则由上述过程可得递推式$h_n=\sum_{k=2}^{n-1}h_k\cdot h_{n-k+1} \quad (n\geq 4)$\par 
    设$g_{n-2}=h_n$,则有初始值$g_0=g_1=1$即递推式$g_n=\sum_{k=0}^{n-1}g_k\cdot g_{n-k-1}$,设其生成函数为$G(x)$,则:
    \begin{align*}
        G(x)&=1+\sum_{n\geq 1}g_nx^n\\
        \because G^2(x)&=\sum_{n\geq 1}x^{n-1}(\sum_{k\geq 0}^{n-1}g_k\cdot g_{n-k-1})
        \therefore G(x)=1+xG^2(x);
    \end{align*}
    解得:$G(x)=\frac{1-\sqrt{1-4x}}{2x}$(通过代入$G(0)=1$可将另解舍去)\par 
    级数展开可得$G(x)=\sum_{n\geq 0}\frac{1}{n+1}\binom{2n}{n}x^n$\par 
    于是$g_n=\frac{1}{n+1}\binom{2n}{n},\quad h_n=g_{n-2}=\frac{1}{n-1}\binom{2n-4}{n-2}$
\end{solution}
    
\begin{question}
    一个圆上等间隔地选出$2n$个点,将这些点用$n$条互不相交的线段连成$n$对,共有多少种不同的方式?
\end{question}

\begin{solution}
    将圆上点顺时针排序为$a_1,a_2,\cdots,a_{2n}$设$f_{2n}$表示圆上2n个点的连线方式,则显然有初始值$f_2=1$\par 
    取圆上任意两点连线分割,不妨记为$(a_1,a_k)$,则该连线将圆上剩余2n-2个点分为两部分,注意到,对任意合法的连线方式,均要求这两部分点均为偶数个,
    (奇数个则必然可以在两侧分别取一个点,使其连线与分割线相交)。\par 
    于是可设一侧点数为$2i\quad (0\leq i \leq n-1)$,则另一侧点数为$2(n-1-i)$,由于两部分连线各不相交,
    对一个确定的分割,圆上2n个点的连线方式即等于圆上2i个点的连线方式和圆上2(n-i)个点的连线方式乘积;\par 
    补充定义$f_0=1$于是有递推式:
    \[f_{2n}=\sum_{i=0}^{n-1}f_{2i}\cdot f_{2(n-1-i)}\quad (n\geq 2)\]\par 
    令$h_n=f_{2n}$,则$h_n$初始值及递推式即为标准的Catalan初始值及递推式,同题5.c中关于$g_n$求解过程可得:$f_{2n}=h_n=\frac{1}{n+1}\binom{2n}{n}$
\end{solution}

\begin{question}
    确定$S_n$中$132$禁位排列的个数,即:排列$\pi$中不存在$3$个位置$i<j<k$满足$\pi(i)<\pi(k)<\pi(j)$。例如,$n=3$时共有$123,213,231,312,321$这$5$种。
\end{question}

\begin{solution}
    首先证明132禁位排列条件等价于一个$a_n,\cdots,a_k,\cdots,a_1$的入栈序列经过一个栈可能生成的出栈序列\par 
    注意到:任意三个入栈元素的出栈排序只与其相对入栈序列有关,与其余入栈元素无关,因此,依据栈“先进后出”的特点,对一个以$a_k,a_j,a_j$顺序
    入栈的序列,可得所有出栈可能的相对顺序为:
    \[a_i,a_j,a_k\quad a_j,a_i,a_k\quad a_j,a_k,a_i\quad a_k,a_i,a_j\quad a_k,a_j,a_i\]\par 
    即所有经过栈的可能排列等价于132禁位排列,于是原题可描述为:确定一个$a_n,\cdots,a_k,\cdots,a_1$的入栈序列经过一个栈可能生成的出栈序列数(记为$h_n$)\par 
    显然有$h_1=1,h_2=2$,设后于$a_n$出栈的共有k个$(0\leq k\leq n-1)$,注意到由于$a_n$最先入栈,当$a_n$出栈时,栈完全清空,因此后于$a_n$出栈的元素为$a_1,\cdots,a_k$,先于$a_n$出栈的元素
    为$a_k,\cdots, a_{n-1}$,且两部分排列数均满足132禁位排列,于是有递推式$h_n=\sum_{k=0}^{n-1}h_k\cdot h_{n-1-k}$其递推式及初始值即为标准Catalan数初始值及递推式,
    同题5.c中关于$g_n$求解过程可得:$h_n=\frac{1}{n+1}\binom{2n}{n}$
\end{solution}

\begin{question}
    有$n$个木块,每块长度为 1(不考虑厚度),质量相同。把它们一个一个垒在一个桌子上,如下图所示,每层只放一个木块。在所有木块保持稳定的前提下,最顶层木块的最右端与桌子边缘的距离最多是多少?
\end{question}

\begin{solution}
    (利用贪心算法求解)不妨将木块自上而下编号为$1,2,\cdots,n$,设木块质量为G,第k个木块右端相对于第k+1个木块右端的伸出量为$x_k$,前k个木块公共重心距第k个木块左端为$g_k$,则显然有$x_1=\frac{1}{2},g_1=\frac{1}{2}$\par 
    考虑k个木块$(k\geq 2)$的情形,首先对前k-1个木块,为保持平衡,应有前k-1个木块公共重心位于第k个木块右端点,于是由力矩公式有:$g_{k}=\frac{\frac{1}{2}\cdot G+1\cdot (k-1)G}{kG}=1-\frac{1}{2k}$\par 
    为使得总伸出量最大,令前k个木块的公共重心位于第k+1个木块右侧,则有第k个木块右端相对于第k+1个木块右端的伸出量为$x_k=1-g_k=\frac{1}{2k}$\par 
    如此进行下去,最终将前n个木块的公共重心放置于桌面右端,则有总伸出量为$S=\frac{1}{2}+\frac{1}{4}+\cdots+\frac{1}{2n}=\frac{1}{2}\sum_{k=1}^n\frac{1}{k}$
\end{solution}
\end{document}