\documentclass[fontset=windows,11pt]{article}
\usepackage[a4paper]{geometry}
\geometry{left=2.0cm,right=2.0cm,top=2.5cm,bottom=2.5cm}

\usepackage[UTF8]{ctex}
\usepackage{amsmath,amsfonts,graphicx,amssymb,bm,amsthm}
\usepackage[noend]{algpseudocode}
\usepackage{mathtools}
\newtheorem{question}{\hskip 1.7em}

\newenvironment{solution}{{\noindent\hskip 2em \bf 解 \quad}}

\renewenvironment{proof}{{\noindent\hskip 2em \bf 证明 \quad}}{\hfill$\qed$\par\vskip1em}
\newcommand\E{\mathbb{E}}

\begin{document}
    \begin{center}
        {\Large \bf 组合数学Homework5}\\
    \end{center}
    \begin{kaishu}
        \hfill 提交者:游昆霖 \quad 学号:2020K8009926006
    \end{kaishu}

    \begin{question}
        设$p(n)$表示分拆数,试比较$p(n)-p(n-1)$和$p(n-1)-p(n-2)$的大小。
    \end{question}

    \begin{solution}
        由于$p(n)-p(n-1)$表示n不含1的分拆数,故有:
    \begin{align*}
        p(n)-p(n-1)&=\sum_{m=1}^{[\frac{n}{2}]}p(n-m,m)\\
        p(n-1)-p(n-2)&=\sum_{m=1}^{[\frac{n-1}{2}]}p(n-1-m,m)
    \end{align*}\par 
    二者相减则有:
    \begin{align*}
        [p(n)-p(n-1)]-[p(n-1)-p(n-2)]\geq \sum_{m=1}^{[\frac{n-1}{2}]}[p(n-m,m)-p(n-1-m,m)]\qquad\text{(1)}
    \end{align*}\par 
    注意到,任意一个n-1-m的m分拆,只需在最大的分拆部分(即$x_{1}$)处加1,即得到了一个n-m的m分拆,因此有$p(n-m,m)\geq p(n-1-m,m)$;\par 
    求和即得$p(n)-p(n-1)\geq p(n-1)-p(n-2)\quad n\geq 3$
    \end{solution}

    \begin{question}
        设$p(n,m)$表示恰好有$m$个正整数组成的$n$的分拆个数,$q(n,m)$表示$n$的分拆中最大的数恰好是$m$的分拆个数。
        \begin{itemize}
            \item[a.] 证明:$p(n,m)=q(n,m)$;
            \item[b.] 证明:$P(x)=\sum_{n\ge 0}p(n,m)x^n=x^m\prod_{k=1}^m\frac{1}{1-x^k}$。提示:利用上一问的结论。
        \end{itemize}
    \end{question}

    \begin{proof}
        a.考虑Ferrers diagram,由于恰好有$m$个正整数组成的$n$的分拆与格子数为n,列数为m的Ferrers diagram一一对应,自底向上,将第k行作为n的分拆中第k个数,则每个Ferrers diagram均与
        $n$的分拆中最大的数恰好是$m$的分拆一一对应。故有$p(n,m)=q(n,m)$,事实上,左端一个分拆与右端一个分拆一一对应,其Ferrers diagram互相共轭。\par 
        b. 考虑将分拆表示为坐标格式,$k_i$表示分拆中i出现次数,则有:
        \begin{align*}
            q(n,m)=\#\{(k_1,k_2,\cdots,k_m)|1\cdot k_1+2\cdot k_2 +\cdots+m\cdot k_m=n,k_i\geq 0,i=1,2,\cdots,m-1,k_m>0\}
        \end{align*}
        由(a)结论有$p(n,m)=q(n,m)$,故
        \begin{align*}
            P(x)&=\sum_{n\geq 0}p(n,m)x^n=\sum_{n\geq 0}q(n,m)x^n\\
                &=(1+x+x^2+\cdots+x^{k_1}+\cdots)\times(1+x^2+x^4+\cdots+x^{2k_2}+\cdots)\times\cdots\\
                &\quad \times(x^m+x^{2m}+\cdots+x^{mk_m}+\cdots)\\
                &=x^m\prod_{k=1}^m\frac{1}{1-x^k}
        \end{align*}
    \end{proof}

    \begin{question}
        证明:$\left\lceil\frac{n}{m}\right\rceil=\left\lfloor\frac{n+m-1}{m}\right\rfloor$。
    \end{question}

    \begin{proof}
        首先有n,m均为正整数,进行分类讨论如下:\par 
        a.若$m|n$,即$n=km,k\in\mathbb{N}$, 等式左边$=k$,等式右边$=\rfloor k+1-\frac{1}{m}\lfloor=k$,成立;\par 
        b.若$m\nmid m$,即$n=km+t,k,t\in\mathbb{N}$则有:
        \begin{align*}
            \left\lceil\frac{n}{m}\right\rceil-\left\lfloor\frac{n+m-1}{m}\right\rfloor
            &=\frac{n}{m}-\{\frac{t}{m}\}+1-\frac{n+m-1}{m}+\{\frac{t-1}{m}\} \\
            &=-\frac{t}{m}+\frac{1}{m}+\frac{t-1}{m}=0   
        \end{align*}\par 
        综上可得:等式恒成立。
    \end{proof}

    \begin{question}
        证明:$\left\lfloor x\right\rfloor+\left\lfloor x+\frac{1}{2}\right\rfloor=\left\lfloor 2x\right\rfloor$。
    \end{question}

    \begin{proof}
        设$\lfloor x\rfloor=k$,进行分类讨论如下:\par 
        a.若$k\leq x< k+\frac{1}{2}$,则有$k+\frac{1}{2}\leq x< k+1\quad 2k\leq 2x< 2k+1$\par 
        故等式左边$=k+k=2k$,等式右边$=2k$,成立\par 
        b.若$k+\frac{1}{2}\leq x< k+1$,则有$k+1\leq x< k+\frac{3}{2}\quad 2k+1\leq 2x< 2k+2$\par 
        故等式左边$=k+k+1=2k+1$,等式右边$=2k+1$,成立\par
        综上可得:等式恒成立。
    \end{proof}

    \begin{question}
        试计算$2^{2021} \mod 11$。
    \end{question}

    \begin{solution}
        由于2和11互素,由费马小定理可得:$2^{10}\equiv 1(mod 11)$;
        因$2021=202*10+1$,故$2^{2021}\equiv 1\times 2^{1}\equiv 2(mod 11)$;
    \end{solution}

    \begin{question}
        证明:任给$m,n\in\mathbb{N}$,都有$m!n!(m+n)! | (2m)!(2n)!$。
    \end{question}

    \begin{proof}
        记$\alpha_p(n):=max\{s|p^s|n\}$,则$\alpha_p(n)=[\frac{n}{p}]+[\frac{n}{p^2}]+\cdots+[\frac{n}{p^s}]$,\par 
        且有$\frac{(2m)!(2n)!}{m!n!(m+n)!}=\prod_{p\quad prime}p^{\alpha_p(2m)+\alpha_p(2n)-\alpha_p(m)-\alpha_p(n)-\alpha_p(m+n)}$\par 
        首先证明引理:$[2a]+[2b]\geq [a]+[b]+[a+b]$\par 
        设$a=m+x,b=n+y,m,n\in \mathbb{Z}, 0\leq x,y<1$,则$[2a]+[2b]=2m+2n+[2x]+[2y],\quad [a]+[b]+[a+b]=2m+2n+[x+y]$\par 
        $\textcircled{1}$当x和y均小于$\frac{1}{2}$时,有$[2x]+[2y]=[x+y]=0$,不等式成立;\par 
        $\textcircled{2}$当x或y有一者大于等于$\frac{1}{2}$时,有$[2x]+[2y]\geq [x]+[y]+1\geq [x+y]$\par 
        综上,引理证毕。\par 
        应用引理立得:$\forall p(prime)\forall r\in \mathbb{N}, [\frac{2m}{p^r}]+[\frac{2n}{p^r}]-[\frac{m}{p^r}]-[\frac{n}{p^r}]-[\frac{m+n}{p^r}]\geq 0$\par 
        对r从1到无穷求和可得:$\alpha_p(2m)+\alpha_p(2n)-\alpha_p(m)-\alpha_p(n)-\alpha_p(m+n)$\par 
        由p任意性,则有$\frac{(2m)!(2n)!}{m!n!(m+n)!}$的素数分解式中,素因子指数均非负,也即$\frac{(2m)!(2n)!}{m!n!(m+n)!}$为整数。\par 
        注记:事实上$\frac{(2m)!(2n)!}{m!n!(m+n)!}$为二元卡特兰数$C(m,n)$,由组合意义显然为整数。
    \end{proof}

    \begin{question}
        利用二次互反律,计算Legendre符号:$\left(\frac{20}{67}\right )$。
    \end{question}
 
    \begin{solution}
        $(\frac{20}{67})=(\frac{5\times 2^2}{67})=(\frac{5}{67})$\par 
        由二次互反律有:$(\frac{5}{67})(\frac{67}{5})=(-1)^{\frac{4\times 66}{4}}=1$\par 
        又因为$(\frac{67}{5})=(\frac{2}{5})=-1$(此因$5=8\times 1-3$);\par 
        故$(\frac{5}{67})=-1$
    \end{solution}

    \begin{question}
        设$p$是奇素数,计算Legendre符号:$\left(\frac{3}{p}\right )$。
    \end{question}

    \begin{solution}
        \begin{equation*}
        \text{由二次互反律有:}(\frac{3}{p})(\frac{p}{3})=(-1)^{\frac{p-1}{2}}=\begin{cases}
            1,\quad p=4k+1\\
            -1,\quad p=4k+3\\
        \end{cases}
    \end{equation*}
    \begin{align*}
        \text{又}\because (\frac{p}{3})&=\begin{cases}
            (\frac{1}{3})=1,\quad p=3t+1\\
            (\frac{2}{3})=-1,\quad p=3t+2\\
        \end{cases}\\
        \therefore (\frac{3}{p})&=\begin{cases}
            1,\quad p=12k\pm 1\\
            -1,\quad p=12k\pm 5\\
        \end{cases} \text{p为奇素数}
    \end{align*}      
    \end{solution}

\end{document}